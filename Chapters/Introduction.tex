\chapter*{Introduction}\label{chap-introduction}

It is a peculiar fact that the entire observable universe can be described by just three fundamental particles: the up and down quarks, and the electron. We must then beg the question as to why it is that we observe three generations of quarks and leptons, where subsequent generations are much heavier than the first. We call this collection of the most fundamental particles the Standard Model (SM) of particle physics. Ever since its first construction, some 50 years ago, it has stood the test of time and held strong against intense scrutiny. The most massive of the fundamental particles is the top quark, with a mass of around $173.3 \GeVcc$, which is () times more massive than the up and down quarks and the heaviest of the fundamental particles by a long shot. 

With the new energy frontier reached by the LHC an abundance of top quarks are produced in the hard collisions produced primarily in the ATLAS and CMS discovery experiments. This level of top production has not been achievable at any other collider experiment, such as the Tevatron at Fermilab, and thus the LHC obtains its title as a top factory. This large production of top quarks allows for extremely precise measurements of the properties of the top quark, such as production cross-section, mass, couplings, spin correlations, forward-backward asymmetry, and charge measurements. The couplings of the top quark are of particular interest due to the fact that the top quark does not hadronise, and can thus be accessed directly. The production cross-section of a particular top decay can be measured and properties inferred from the energy spectrum. Both of these techniques are the focus of this analysis. 

Despite the discovery of the Higgs boson on the 4th of July, 2012, by the ATLAS and CMS experiments \cite{ATLASHiggs, CMSHiggs} at the Large Hadron Collider, top physics analyses remain some of the highest priority physics analyses at the LHC. Top pair production and mass are essential precision measurements due to the direct link of the top quark with the Higgs mass. This can be seen through the Yukawa coupling of the top, such that it is extremely close to unity, and implies that the fine tuning of the Higgs mass is dependent on its coupling to the top. One of the most important features of the top quark is that its signature is a primary background to many new physics processes beyond that of the Standard Model, where most models beyond the Standard Model extend the top sector, thus introducing more degrees of freedom and solutions to known problems (such as the hierarchy problem).

In Chapter \ref{chap-theory} we discuss the composition of the Standard Model and the symmetry groups that it is based on, a much more detailed explanation of how particles acquire mass, and an in-depth description of the top quark. We will focus on how couplings of the top quark to a gauge boson are constructed, and what implications this might have should we see a deviation from what is predicted by theory. Chapter \ref{chap-detector} describes the design and performance of the CMS detector, with an emphasis on the electromagnetic calorimeter which is of high importance for the $t\bar{t}+\gamma$ analysis due to the strong ability to reconstruct photons. 

Chapter \ref{chap-EventReconstruction&Simulation} begins the event simulation and reconstruction section of the analysis, describing the way in which we identify particles within the detector and transfer the output into data in which offline analysis is performed. This chapter also describes the process for the simulation of our official CMS signal sample and comparison to other Monte Carlo event generators. The second part of the analysis is described in Chapter \ref{chap-EventSelection}, where we state the selection process for our signal events. We break this down into two categories: top quark pair selection, and photon selection. This method for selection allows us to calculate selection efficiencies in much more convenient manner. We then describe the process for the estimation of the number of background events within our selection. 

After we have calculated our event yields, taking into account background processes, in Chapter \ref{chap-crosssection} we calculate the production cross-section for top quarks with a radiated photon, decaying to final states containing two oppositely-signed leptons and at least two jets (where one is b-tagged). This is broken down into several variables which we calculate separately. Due to the way in which we perform the analysis, the way in which objects are reconstructed and the detector is composed, we must correct for such effects by calculating systematic and statistical uncertainties. These are described in Chapter \ref{chap-SystematicUncertainties}, where we calculate all individual uncertainties and incorporate them into the final cross-section measurement. 

% the observation of Neutrino oscillations \cite{} in the Tokai to Kamioka (T2K) experiment

% we can measure the way in which enormous stellar masses, such as planets, affect the motion of other high mass objects. 
% we know so little about gravity. Very recently we 

% the first observation of gravitation waves produced by a binary black-hole system merging into one by the LIGO experiment in 2016 \cite{PhysRevLett.116.061102}