\chapter{Results} \label{chap-Results}

The values used for measuring the cross section ratio can be found in Table \ref{tab-xsectvariables}.

\begin{table}[h!]
\begin{center}
\begin{tabular}{l|c|c|c}
\hline
\hline
	\textbf{Value} & $\mu^+\mu^-$ & $e^+e^-$ & $e\mu$ \\
\hline
	Number of signal events, $N_{signal}$ & & & \\
	$t\bar{t}+\gamma$ Top Selection $\epsilon^{t\bar{t}+\gamma}_{top} \cdot A^{t\bar{t}+\gamma}_{top}$ & & & \\
	$t\bar{t}+\gamma$ Photon Selection $\epsilon^{t\bar{t}+\gamma}_{\gamma} \cdot A^{t\bar{t}+\gamma}_{\gamma}$ & & & \\
	$t\bar{t}$ Number of $t\bar{t}$ events $N^{t\bar{t}}$ & & & \\
	$t\bar{t}$ Top Selection $\epsilon^{t\bar{t}}_{top} \cdot A^{t\bar{t}}_{top}$ & & & \\
	$t\bar{t}+\gamma$ Visible Photon Selection $\epsilon^{t\bar{t}+\gamma Vis}_{\gamma} \cdot A^{t\bar{t}+\gamma Vis}_{\gamma}$ & & & \\
\hline
	Total & & & \\
\hline
\hline
\end{tabular} 
\end{center}
\caption{Values used for calculating the cross section ratios in the $\mu^+\mu^-$, $e^+e^-$, and $e\mu$ channels.}
\label{tab-xsectvariables}
\end{table}

We calculate the cross section ratios for each of the dilepton channels to be:

\begin{equation}
	R_{\mu^+\mu^-} = \frac{\sigma_{t\bar{t}+\gamma}}{\sigma_{t\bar{t}}} = \frac{N_{signal}}{\epsilon_{t\bar{t}+\gamma}} \cdot \frac{\epsilon^{t\bar{t}}_{top} A^{t\bar{t}}_{top}}{N_{t\bar{t}}} = 
\end{equation}

\begin{equation}
	R_{e^+e^-} = \frac{\sigma_{t\bar{t}+\gamma}}{\sigma_{t\bar{t}}} = \frac{N_{signal}}{\epsilon_{t\bar{t}+\gamma}} \cdot \frac{\epsilon^{t\bar{t}}_{top} A^{t\bar{t}}_{top}}{N_{t\bar{t}}} = 
\end{equation}

\begin{equation}
	R_{e\mu} = \frac{\sigma_{t\bar{t}+\gamma}}{\sigma_{t\bar{t}}} = \frac{N_{signal}}{\epsilon_{t\bar{t}+\gamma}} \cdot \frac{\epsilon^{t\bar{t}}_{top} A^{t\bar{t}}_{top}}{N_{t\bar{t}}} =  
\end{equation}

Another result is visible cross section ratio where we do not extrapolate the measured result to the phase space used for signal simulation. In this way we do not rely on kinematic properties of simulated signal dataset. We will need photon reconstruction and identification efficiency, but not signal acceptance for photon selection. Efficiency is calculated as ratio of number of generated signal events that passed photon selection to number of generated signal events with a generator level signal photon in the region of p$_T - \eta$ space used for photon selection (the same p$_T$ and $\eta$ cuts are applied to generated photons as used for the reconstructed photon). The visible cross section ratio is measured in each respective dilepton channel to be:

\begin{equation}
	R_{\mu^+\mu^-} = \frac{\sigma_{t\bar{t}+\gamma}}{\sigma_{t\bar{t}}} = \frac{N_{signal}}{\epsilon_{t\bar{t}+\gamma}} \cdot \frac{\epsilon^{t\bar{t}}_{top} A^{t\bar{t}}_{top}}{N_{t\bar{t}}} = 
\end{equation}

\begin{equation}
	R_{e^+e^-} = \frac{\sigma_{t\bar{t}+\gamma}}{\sigma_{t\bar{t}}} = \frac{N_{signal}}{\epsilon_{t\bar{t}+\gamma}} \cdot \frac{\epsilon^{t\bar{t}}_{top} A^{t\bar{t}}_{top}}{N_{t\bar{t}}} = 
\end{equation}

\begin{equation}
	R_{e\mu} = \frac{\sigma_{t\bar{t}+\gamma}}{\sigma_{t\bar{t}}} = \frac{N_{signal}}{\epsilon_{t\bar{t}+\gamma}} \cdot \frac{\epsilon^{t\bar{t}}_{top} A^{t\bar{t}}_{top}}{N_{t\bar{t}}} = 
\end{equation}

\begin{table}
\centering
\begin{tabular}{|l|l|}
\hline
	\textbf{Parameter} & \textbf{Value} \\
\hline
	$t\bar{t}+\gamma events$ & \\
	$t\bar{t}+\gamma eff.$ & \\
	$t\bar{t} events$ & \\
	$t\bar{t} eff$ & \\
\hline
\end{tabular}
\caption{}
\end{table}	

\section{Combination of channels} \label{sec-CombinationOfChannels}

In order to improve our measurement, we combine each decay channel to produce a single result for the cross-section ratio. This is done by using a single likelihood fit, similar to that seen previously in Section \ref{sec-SigEventsInData}. The incorporates nine input parameters: the top purity, photon purity, and number of data events for each dilepton decay channel while fitting to the same three scale factors. We therefore define a modified chi-squared calculation to be the sum of the nine terms, given as:

\begin{equation}
\begin{split}
\chi^2(SF_{t\bar{t}+\gamma},SF_{V+\gamma},SF_{jet\to\gamma}) & = \frac{\left(\pi^{data,\mu^+\mu^-}_{e\gamma} - \pi^{MC,\mu^+\mu^-}_{e\gamma}\right)^2}{\sigma^2_{\pi_{e\gamma},\mu^+\mu^-}} + \frac{\left(\pi^{data,\mu^+\mu^-}_{t\bar{t}} - \pi^{MC,\mu^+\mu^-}_{t\bar{t}}\right)^2}{\sigma^2_{\pi_{t\bar{t}},\mu^+\mu^-}} + \frac{\left(N^{data,\mu^+\mu^-}_{events} - N^{MC,\mu^+\mu^-}_{events}\right)^2}{\sigma^2_{N_{events,\mu^+\mu^-}}} \\
& + \frac{\left(\pi^{data,e^+e^-}_{e\gamma} - \pi^{MC,e^+e^-}_{e\gamma}\right)^2}{\sigma^2_{\pi_{e\gamma},e^+e^-}} + \frac{\left(\pi^{data,e^+e^-}_{t\bar{t}} - \pi^{MC,e^+e^-}_{t\bar{t}}\right)^2}{\sigma^2_{\pi_{t\bar{t}},e^+e^-}} + \frac{\left(N^{data,e^+e^-}_{events} - N^{MC,e^+e^-}_{events}\right)^2}{\sigma^2_{N_{events,e^+e^-}}} \\
& + \frac{\left(\pi^{data,e\mu}_{e\gamma} - \pi^{MC,e\mu}_{e\gamma}\right)^2}{\sigma^2_{\pi_{e\gamma},e\mu}} + \frac{\left(\pi^{data,e\mu}_{t\bar{t}} - \pi^{MC,e\mu}_{t\bar{t}}\right)^2}{\sigma^2_{\pi_{t\bar{t}},e\mu}} + \frac{\left(N^{data,e\mu}_{events} - N^{MC,e\mu}_{events}\right)^2}{\sigma^2_{N_{events,e\mu}}} 
\end{split}
\end{equation}

The values of the top purity, photon purity, and number of events in data are the same as used previously from individual channels. The values are listed in Table \ref{tab-likelihoodVariables}.  

\begin{figure}
\includegraphics[width=0.3\textwidth]{Plots/likelihoodfits/Combination/ttgamma_likelihood_fit.pdf}
\includegraphics[width=0.3\textwidth]{Plots/likelihoodfits/Combination/vgamma_likelihood_fit.pdf}
\includegraphics[width=0.3\textwidth]{Plots/likelihoodfits/Combination/jgamma_likelihood_fit.pdf}
\caption{Likelihood distributions for the $t\bar{t}+\gamma$, $V+\gamma$, and $jets\to \gamma$ scale factors when combining channels.}
\label{fig-SFLikelihoodFitsCombination}
\end{figure}