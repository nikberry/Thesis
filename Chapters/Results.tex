\chapter{Results} \label{chap-Results}

The values used for measuring the cross section ratio can be found in Table \ref{tab-xsectvariables}.

\begin{table}[h!]
\begin{center}
\begin{tabular}{l|c|c|c}
\hline
\hline
	\textbf{Value} & $\mu^+\mu^-$ & $e^+e^-$ & $e\mu$ \\
\hline
	Number of signal events, $N_{signal}$ & & & \\
	$t\bar{t}+\gamma$ Top Selection $\epsilon^{t\bar{t}+\gamma}_{top} \cdot A^{t\bar{t}+\gamma}_{top}$ & & & \\
	$t\bar{t}+\gamma$ Photon Selection $\epsilon^{t\bar{t}+\gamma}_{\gamma} \cdot A^{t\bar{t}+\gamma}_{\gamma}$ & & & \\
	$t\bar{t}$ Number of $t\bar{t}$ events $N^{t\bar{t}}$ & & & \\
	$t\bar{t}$ Top Selection $\epsilon^{t\bar{t}}_{top} \cdot A^{t\bar{t}}_{top}$ & & & \\
	$t\bar{t}+\gamma$ Visible Photon Selection $\epsilon^{t\bar{t}+\gamma Vis}_{\gamma} \cdot A^{t\bar{t}+\gamma Vis}_{\gamma}$ & & & \\
\hline
	Total & & & \\
\hline
\hline
\end{tabular} 
\end{center}
\caption{Values used for calculating the cross section ratios in the $\mu^+\mu^-$, $e^+e^-$, and $e\mu$ channels.}
\label{tab-xsectvariables}
\end{table}

We calculate the cross section ratios for each of the dilepton channels to be:

\begin{equation}
	R_{\mu^+\mu^-} = \frac{\sigma_{t\bar{t}+\gamma}}{\sigma_{t\bar{t}}} = \frac{N_{signal}}{\epsilon^{t\bar{t}+\gamma}_{top} A^{t\bar{t}+\gamma}_{top} \epsilon^{t\bar{t}+\gamma}_{\gamma} A^{t\bar{t}+\gamma}_{\gamma}} \cdot \frac{\epsilon^{t\bar{t}}_{top} A^{t\bar{t}}_{top}}{N_{t\bar{t}}} = 
\end{equation}

\begin{equation}
	R_{e^+e^-} = \frac{\sigma_{t\bar{t}+\gamma}}{\sigma_{t\bar{t}}} = \frac{N_{signal}}{\epsilon^{t\bar{t}+\gamma}_{top} A^{t\bar{t}+\gamma}_{top} \epsilon^{t\bar{t}+\gamma}_{\gamma} A^{t\bar{t}+\gamma}_{\gamma}} \cdot \frac{\epsilon^{t\bar{t}}_{top} A^{t\bar{t}}_{top}}{N_{t\bar{t}}} = 
\end{equation}

\begin{equation}
	R_{e\mu} = \frac{\sigma_{t\bar{t}+\gamma}}{\sigma_{t\bar{t}}} = \frac{N_{signal}}{\epsilon^{t\bar{t}+\gamma}_{top} A^{t\bar{t}+\gamma}_{top} \epsilon^{t\bar{t}+\gamma}_{\gamma} A^{t\bar{t}+\gamma}_{\gamma}} \cdot \frac{\epsilon^{t\bar{t}}_{top} A^{t\bar{t}}_{top}}{N_{t\bar{t}}} = 
\end{equation}

Another result is visible cross section ratio where we do not extrapolate the measured result to the phase space used for signal simulation. In this way we do not rely on kinematic properties of simulated signal dataset. We will need photon reconstruction and identification efficiency, but not signal acceptance for photon selection. Efficiency is calculated as ratio of number of generated signal events that passed photon selection to number of generated signal events with a generator level signal photon in the region of p$_T - \eta$ space used for photon selection (the same p$_T$ and $\eta$ cuts are applied to generated photons as used for the reconstructed photon). The visible cross section ratio is measured in each respective dilepton channel to be:

\begin{equation}
	R^{Vis.}_{\mu^+\mu^-} = \frac{\sigma_{t\bar{t}+\gamma}}{\sigma_{t\bar{t}}} = \frac{N_{signal}}{\epsilon^{t\bar{t}+\gamma Vis.}_{top} A^{t\bar{t}+\gamma Vis.}_{top} \epsilon^{t\bar{t}+\gamma Vis.}_{\gamma} A^{t\bar{t}+\gamma Vis}_{\gamma}} \cdot \frac{\epsilon^{t\bar{t}}_{top} A^{t\bar{t}}_{top}}{N_{t\bar{t}}} = 
\end{equation}

\begin{equation}
	R^{Vis.}_{e^+e^-} = \frac{\sigma_{t\bar{t}+\gamma}}{\sigma_{t\bar{t}}} = \frac{N_{signal}}{\epsilon^{t\bar{t}+\gamma Vis.}_{top} A^{t\bar{t}+\gamma Vis.}_{top} \epsilon^{t\bar{t}+\gamma Vis.}_{\gamma} A^{t\bar{t}+\gamma Vis}_{\gamma}} \cdot \frac{\epsilon^{t\bar{t}}_{top} A^{t\bar{t}}_{top}}{N_{t\bar{t}}} = 
\end{equation}

\begin{equation}
	R^{Vis.}_{e\mu} = \frac{\sigma_{t\bar{t}+\gamma}}{\sigma_{t\bar{t}}} = \frac{N_{signal}}{\epsilon^{t\bar{t}+\gamma Vis.}_{top} A^{t\bar{t}+\gamma Vis.}_{top} \epsilon^{t\bar{t}+\gamma Vis.}_{\gamma} A^{t\bar{t}+\gamma Vis}_{\gamma}} \cdot \frac{\epsilon^{t\bar{t}}_{top} A^{t\bar{t}}_{top}}{N_{t\bar{t}}} = 
\end{equation}

\begin{table}
\centering
\begin{tabular}{|l|l|}
\hline
	\textbf{Parameter} & \textbf{Value} \\
\hline
	$t\bar{t}+\gamma events$ & \\
	$t\bar{t}+\gamma eff.$ & \\
	$t\bar{t} events$ & \\
	$t\bar{t} eff$ & \\
\hline
\end{tabular}
\caption{}
\end{table}	