\chapter{Results} \label{chap-Results}

Now that we have successfully computed the observables that we are interested in, we can calculte the cross-section and ratio of $t\bar{t}+\gamma$ and inclusive $t\bar{t}$ cross-sections, $R = \sigma_{t\bar{t}+\gamma}/\sigma_{t\bar{t}}^{incl.}$, including systematic uncertainties. The values of the variables used in calculating the cross-section ratio can be found in Table \ref{tab-xsectvariables}.

\begin{table}[h!]
\begin{center}
\begin{tabular}{l|c|c|c}
\hline
	\textbf{Value} & $\mu^+\mu^-$ & $e^+e^-$ & $e\mu$ \\
\hline
	Number of signal events, $N_{signal}$ & & & \\
	$t\bar{t}+\gamma$ Top Selection Efficiency $\epsilon^{t\bar{t}+\gamma}_{top}$ & & & \\
	$t\bar{t}+\gamma$ Photon Selection Efficiency $\epsilon^{t\bar{t}+\gamma}_{\gamma}$ & & & \\
	Number of $t\bar{t}$ events $N^{t\bar{t}}$ & & & \\
	$t\bar{t}$ Top Selection $\epsilon^{t\bar{t}}_{top} \cdot A^{t\bar{t}}_{top}$ & & & \\
\hline
	Total & & & \\
\hline
\end{tabular} 
\end{center}
\caption{Values used in calculating the cross-section ratios in the $\mu^+\mu^-$, $e^+e^-$, and $e\mu$ channels.}
\label{tab-xsectvariables}
\end{table}

We thererfore use the variables to calculate the cross-section ratios for each of the dilepton channels, defined as:

\begin{equation}
	R_{\mu^+\mu^-} = \frac{\sigma_{t\bar{t}+\gamma}}{\sigma_{t\bar{t}}} = \frac{N_{signal}}{\epsilon_{t\bar{t}+\gamma}} \cdot \frac{\epsilon^{t\bar{t}}_{top} A^{t\bar{t}}_{top}}{N_{t\bar{t}}} = 0.00067 \pm 0.00021 (stat. + syst.).
\end{equation}

\begin{equation}
	R_{e^+e^-} = \frac{\sigma_{t\bar{t}+\gamma}}{\sigma_{t\bar{t}}} = \frac{N_{signal}}{\epsilon_{t\bar{t}+\gamma}} \cdot \frac{\epsilon^{t\bar{t}}_{top} A^{t\bar{t}}_{top}}{N_{t\bar{t}}} = 0.00064 \pm 0.00019 (stat. + syst.).
\end{equation}

\begin{equation}
	R_{e\mu} = \frac{\sigma_{t\bar{t}+\gamma}}{\sigma_{t\bar{t}}} = \frac{N_{signal}}{\epsilon_{t\bar{t}+\gamma}} \cdot \frac{\epsilon^{t\bar{t}}_{top} A^{t\bar{t}}_{top}}{N_{t\bar{t}}} = 0.00066 \pm 0.00024 (stat. + syst.).
\end{equation}


We can therefore calculate the fiducial cross-sections for individual decay channels. We do this by extrapolating from the cross-section ratio and multiplying our coputed ratio value by the inclusive $t\bar{t}$ cross-section of $245.6 \pm 1.3(stat.) \pm_{-5.5}^{+6.6}(syst.) \pm 6.5(lumi.)$ pb \cite{Khachatryan:2016mqs}. The results of multiplying the ratio by the inclusive $t\bar{t}$ cross-section for individual decay channels are given as:

\begin{equation}
	\sigma_{t\bar{t}+\gamma}^{\mu^+\mu^-} = 342 \pm 45 (stat. + syst.)
\end{equation}

\begin{equation}
	\sigma_{t\bar{t}+\gamma}^{e^+e^-} = 188 \pm 45 (stat. + syst.)
\end{equation}

\begin{equation}
	\sigma_{t\bar{t}+\gamma}^{e\mu} = 312 \pm 45 (stat. + syst.)
\end{equation}

We are then able to calculate the cross-section multiplied by the dilepton branching fraction, $\mathcal{B}_{t\bar{t}\to2l}$, by dividing through by the kinematic acceptance. The kinematic acceptances are given in Section \ref{subsec-SignalAcceptanceCalculation} as $0.2380 \pm 0.0014$, $0.2433 \pm 0.0014$, and $0.2551 ± 0.0014$ in the $\mu^+\mu^-$, $e^+e^-$, and $e\mu$ final states, respectively. Therefore, we calculate the cross-section multiplied by branching fraction as:

\begin{equation}
	\sigma_{t\bar{t}+\gamma}^{\mu^+\mu^-}\times \mathcal{B}_{t\bar{t}\to2l} = 582 \pm 187
\end{equation}

\begin{equation}
	\sigma_{t\bar{t}+\gamma}^{e^+e^-}\times \mathcal{B}_{t\bar{t}\to2l}  = 582 \pm 187
\end{equation}

\begin{equation}
	\sigma_{t\bar{t}+\gamma}^{e\mu}\times \mathcal{B}_{t\bar{t}\to2l}  = 582 \pm 187
\end{equation}

These values are in agreement with the theoretical prediction of $592 \pm 71 (scale) \pm 30 (PDF)$ fb for the cross-section multiplied by the branching fraction of each of the dileptonic decay channel final states \cite{QCDcorrttgamma}.

\section{Combination of channels} \label{sec-CombinationOfChannels}

In order to improve our measurement, we combine each decay channel to produce a single result for the cross-section ratio. This is done by using a single likelihood fit, similar to that seen previously in Section \ref{sec-SigEventsInData}. The fit incorporates nine input parameters: the top purity, photon purity, and number of data events for each dilepton decay channel while fitting to the same three scale factors. We therefore define a modified chi-squared calculation to be the sum of the nine terms, given as:

\begin{equation}
\begin{split}
\chi^2(SF_{t\bar{t}+\gamma},SF_{V+\gamma},SF_{jet\to\gamma}) & = \frac{\left(\pi^{data,\mu^+\mu^-}_{e\gamma} - \pi^{MC,\mu^+\mu^-}_{e\gamma}\right)^2}{\sigma^2_{\pi_{e\gamma},\mu^+\mu^-}} + \frac{\left(\pi^{data,\mu^+\mu^-}_{t\bar{t}} - \pi^{MC,\mu^+\mu^-}_{t\bar{t}}\right)^2}{\sigma^2_{\pi_{t\bar{t}},\mu^+\mu^-}} + \frac{\left(N^{data,\mu^+\mu^-}_{events} - N^{MC,\mu^+\mu^-}_{events}\right)^2}{\sigma^2_{N_{events,\mu^+\mu^-}}} \\
& + \frac{\left(\pi^{data,e^+e^-}_{e\gamma} - \pi^{MC,e^+e^-}_{e\gamma}\right)^2}{\sigma^2_{\pi_{e\gamma},e^+e^-}} + \frac{\left(\pi^{data,e^+e^-}_{t\bar{t}} - \pi^{MC,e^+e^-}_{t\bar{t}}\right)^2}{\sigma^2_{\pi_{t\bar{t}},e^+e^-}} + \frac{\left(N^{data,e^+e^-}_{events} - N^{MC,e^+e^-}_{events}\right)^2}{\sigma^2_{N_{events,e^+e^-}}} \\
& + \frac{\left(\pi^{data,e\mu}_{e\gamma} - \pi^{MC,e\mu}_{e\gamma}\right)^2}{\sigma^2_{\pi_{e\gamma},e\mu}} + \frac{\left(\pi^{data,e\mu}_{t\bar{t}} - \pi^{MC,e\mu}_{t\bar{t}}\right)^2}{\sigma^2_{\pi_{t\bar{t}},e\mu}} + \frac{\left(N^{data,e\mu}_{events} - N^{MC,e\mu}_{events}\right)^2}{\sigma^2_{N_{events,e\mu}}} 
\end{split}
\end{equation}

The values of the top purity, photon purity, and number of events in data are the same as used previously from individual channels. The values are listed in Table \ref{tab-likelihoodVariables}.  

The likelihood fit is scanned over three scale factor parameters, such that the maximum likelihood returns values of $SF_{t\bar{t}+\gamma} = $ , $SF_{V+\gamma}$, and $SF_{jet\to\gamma}$  for the three scale factors, respectively. Figure \ref{fig-SFLikelihoodFitsCombination} shows the negative log likelihood ratio distributions for each scale factor parameter.  

\begin{figure}
\includegraphics[width=0.3\textwidth]{Plots/likelihoodfits/Combination/ttgamma_likelihood_fit.pdf}
\includegraphics[width=0.3\textwidth]{Plots/likelihoodfits/Combination/vgamma_likelihood_fit.pdf}
\includegraphics[width=0.3\textwidth]{Plots/likelihoodfits/Combination/jgamma_likelihood_fit.pdf}
\caption{Likelihood distributions for the $t\bar{t}+\gamma$, $V+\gamma$, and $jets\to \gamma$ scale factors when combining channels.}
\label{fig-SFLikelihoodFitsCombination}
\end{figure}

After the fit has been performed, the combined number of $t\bar{t}+\gamma$ signal events is $106 \pm 12$. We measure the values of the combined efficiencies for selection, and number of $t\bar{t}$ events to be:

\begin{itemize}
	\item $t\bar{t}+\gamma$ Top and Photon Selection Efficiency $= 0.2321 \pm 0.0042$
	\item Top selection efficiency $= 0.122 \pm 0.000034$
	\item Number of $t\bar{t}$ events $= 102455 \pm 1032$
\end{itemize}

Using these values, we can calculate the cross-section ratio for the combination of channels to be:

\begin{equation}
	R_{comb.} = \frac{\sigma_{t\bar{t}+\gamma}}{\sigma_{t\bar{t}}} = \frac{N_{signal}}{\epsilon_{t\bar{t}+\gamma}} \cdot \frac{\epsilon^{t\bar{t}}_{top} A^{t\bar{t}}_{top}}{N_{t\bar{t}}} = 0.00221 \pm 0.00023
\end{equation}

In the same manner as before, we can multiply the computed ratio by the $t\bar{t}$ inclusive cross-section to find the fiducial cross-section to get:

\begin{equation}
	\sigma_{t\bar{t}+\gamma}^{comb.} = 543 \pm 0.00032 (stat.+syst.)
\end{equation}

\begin{table}[h!] 
\centering
\begin{tabular}{|l|c|c|}
\hline
\textbf{Source} & \multicolumn{2}{c|}{\textbf{Uncertainty (\%)}} \\ \cline{2-3}
 & Up & Down \\
\hline
Statistical & & \\
\hline
Systematic & & \\
\hline
Pileup (PU) & & \\
Out-of-time Pileup (OOT) & & \\
Top P$_{\text{T}}$ & & \\
b-tag & & \\
Photon E$_{\text{T}}$ & & \\
JEC & & \\
JER & & \\
Electron Efficiency & & \\
Electron P$_{\text{T}}$ & & \\
PDF & & \\
\hline
Total & & \\
\hline
\end{tabular} 
\caption{Systematic uncertainties and their contribution to the cross-section ratio for the combination of dielpton final states.}
\label{tab-systuncertsCombined}
\end{table}