\chapter*{Conclusions}\label{chap-conclusions}

In this thesis we present the observation and cross-section measurement of top quark pairs in association with a radiated photon decaying to a dilepton final state. The measurement was made using data from proton-proton collisions recorded by the CMS detector at the Large Hadron Collider, running with a centre-of-mass energy of $\sqrt{s} = 8$ TeV over the 2012 data-taking period. 

\section{Future Outlook}

Overall, the outlook for a measurement of the $t\bar{t}+\gamma$ process at higher energies is an exciting prospect. It will be possible to measure the cross-section of the process to a much higher degree of accuracy due to the increase in the production of top quark pairs compared to a much lower production rate of background processes. For an LHC running at centre-of-mass energy of 14 TeV, we expect a cross-section for top quark pairs to have increased by $\sim3.5$ times at $900$ pb \cite{}, compared to an increase of $\sim1.5$ times for background processes at $150$ pb \cite{}. This can be seen in Figure \ref{}. This increase in top quark pair production would remove the main inhibitor of the measurement -- it is statistically limited.

Ultimately, we would like to measure the electromagnetic vertex of the top quark and radiated photon. 

A combination of semileptonic and dileptonic channels would be more beneficial at higher energies due to increased statistics. 

ttH as two photons final state of higgs