\chapter*{Conclusions}\label{chap-conclusions}

In this thesis we present the observation and cross-section measurement of top quark pairs in association with a radiated photon decaying to a dilepton final state. The measurement was made using data from proton-proton collisions recorded by the CMS detector at the Large Hadron Collider, running with a centre-of-mass energy of $\sqrt{s} = 8$ TeV over the 2012 data-taking period. 

A cut-based analysis measuring the production cross-section of top quark pair events in association of a radiated photon, as well as the ratio of the production cross-section to the inclusive top quark pair cross-section, was carried out using the full 2012 dataset corresponding to an integrated luminosity of $19.7$ fb$^{-1}$. We take the ratio of cross-sections in order to cancel out global variables, such as luminosity. The estimation of the photon identification efficiency is calculated by studying the photon isolation profile using the supercluster footprint removal technique and random cone isolation. This method allows us to extract signal and background templates directly from data. The technique is cross-checked with simulated events for completeness. 

The main sources of uncertainty for the measurement manifest in the form of the purity of top pair events passing selection, and the number of photon events passing full selection. The precision of the cross-section calculation is limited due to the very small number of events passing selection. A cross-section of $\sigma_{t\bar{t}+\gamma} = 983 \pm 32$ fb and ratio $R = \sigma_{t\bar{t}+\gamma}/\sigma_{t\bar{t}} = 0.00221 \pm 0.00023$ was measured in comparison to the theoretical prediction of $\sigma_{t\bar{t}+\gamma}^{theoretical} = 861 \pm 182$ fb, using the latest measurement of the inclusive $t\bar{t}$ cross-section. Therefore, we observe a good agreement with the Standard Model and do not observe any evidence for physics beyond that of the Standard Model. This is the most accurate measurement of the $t\bar{t}+\gamma$ process to date and the only measurement in the dilepton final state ever performed.  


\section{Future Outlook}

Overall, the outlook for a measurement of the $t\bar{t}+\gamma$ process at higher energies is an exciting prospect. It will be possible to measure the cross-section of the process to a much higher degree of accuracy due to the increase in the production of top quark pairs compared to a much lower production rate of background processes. For an LHC running at centre-of-mass energy of $\sqrt{s} = 14$ TeV, we expect a cross-section for top quark pairs to have increased by $\sim3.5$ times the cross-section at $\sqrt{s} = 8$ TeV at $920$ pb \cite{Czakon:2013goa}, compared to an increase of $\sim1.5$ times that at $\sqrt{s} = 8$ TeV for background processes. This can be seen in Figure \ref{fig-ttbarXsectPlot}. This increase in top quark pair production would remove the main inhibitor of the measurement -- it is statistically limited.

Ultimately, we would like to measure the electromagnetic vertex of the top quark and radiated photon, however it could also be used in conjunction with other measurements. For example, a future $t\bar{t}+\gamma$ measurement could also be used in a way that is complementary to the search for top quark pair plus a radiated Higgs boson, whereby the Higgs decays to two photons in the final state. Understanding this process will be of huge importance as it will be a background to the $t\bar{t}+\gamma$ process, and a combination of semileptonic and dileptonic channels would be more beneficial at higher energies due to increased statistics. Similarly, understanding the $t\bar{t}+\gamma$ process is greatly import as it is a background to many SUSY processes. At higher energies it will be possible to glean a greater understanding of the process by measuring the differential cross-section with respect to global variables. 
